\documentclass[12pt]{article}
\usepackage{scrtime} % for \thistime (this package MUST be listed first!)
\usepackage[margin=1in]{geometry}
\usepackage{amsmath} % essential for cases environment
\usepackage{amsthm} % for theorems and proofs
\usepackage{amsfonts} % mathbb
\usepackage{graphics,graphicx}
\usepackage{lineno}
\usepackage{color}
\definecolor{aqua}{RGB}{0, 128, 225}
\usepackage[colorlinks=true,citecolor=aqua,linkcolor=aqua,urlcolor=aqua]{hyperref}
%%%%%%%%%%%%%%%%%%%%%%%%%%%%%%%%%%%
%% FANCY HEADER AND FOOTER STUFF %%
%%%%%%%%%%%%%%%%%%%%%%%%%%%%%%%%%%%
\usepackage{fancyhdr,lastpage}
\pagestyle{fancy}
\fancyhf{} % clear all header and footer parameters
%%%\lhead{Student Name: \theblank{4cm}}
%%%\chead{}
%%%\rhead{Student Number: \theblank{3cm}}
%%%\lfoot{\small\bfseries\ifnum\thepage<\pageref{LastPage}{CONTINUED\\on next page}\else{LAST PAGE}\fi}
\lfoot{}
\cfoot{{\small\bfseries Page \thepage\ of \pageref{LastPage}}}
\rfoot{}
\renewcommand\headrulewidth{0pt} % Removes funny header line
%%%%%%%%%%%%%%%%%%%%%%%%%%%%%%%%%%%

\title{The Theory of Everything}

\author{\underline{\emph{Group Name}}: \texttt{{\color{blue}Two Nobels}}\\
{}\\
\underline{\emph{Group Members}}: {\color{blue}Marie Curie, Linus Pauling, John Bardeen, Frederick Sanger}}

\date{\today\ @ \thistime}

\begin{document}
\linenumbers

\maketitle

\begin{abstract}
We solve everything because we're really smart\footnote{Marie Curie won the Nobel prize in 1903 (for Physics) and 1911 (for Chemistry); Linus Pauling in 1954 (for Chemistry) and 1962 (for Peace); John Bardeen in 1956 and 1972 (both for Physics); Frederick Sanger 1958 and 1980 (both for Chemistry).}.
\end{abstract}

\tableofcontents

\section{Background}

It's time for a theory of everything.  Since we're all really smart, we've created one.

\section{Methods}

There is one equation for our theory:
%
\begin{linenomath*}
\begin{equation}\label{E:U}
U = 0 \,.
\end{equation}
\end{linenomath*}
%
We leave it as an exercise for the reader to define $U$.

We exploit Euler's formula,
%
\begin{linenomath*}
\begin{equation}\label{E:Euler}
e^{i\pi} + 1 = 0 \,.
\end{equation}
\end{linenomath*}
%


\section{Results}

\autoref{E:U} follows from \autoref{E:Euler} together with the results in a recent brilliant paper \cite{Dush96b}.

\section{Discussion}

This is really important stuff.

\bibliographystyle{vancouver}
\bibliography{project}

\end{document}
